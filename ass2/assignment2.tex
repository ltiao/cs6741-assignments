\documentclass[10pt, a4paper]{article}
\usepackage{amsmath}
\usepackage{amssymb}
\usepackage{amsthm}
\usepackage{epsfig}
\usepackage[margin=1in]{geometry}
%\usepackage{enumitem}
\usepackage{enumerate}
\usepackage{graphicx}
\usepackage{tikz}
\usetikzlibrary{automata, positioning, shapes}
% \usepackage{pgfplots}
\usepackage{caption}
\usepackage{mathtools}
\usepackage{titling}
\usepackage[lined,boxed,commentsnumbered,vlined]{algorithm2e}

\usepackage{todonotes}

% For writing code:
\usepackage{fancyvrb}
\DefineVerbatimEnvironment{code}{Verbatim}{fontsize=\small}
	\def\showvrb#1{%
	\texttt{\detokenize{#1}}%
}

\usepackage{cleveref}

\linespread{1.1}
\setlength{\parindent}{0pt} % Sets the indent length to 0
\setlength{\parskip}{8pt plus 1pt minus 1pt} % paragraph vertical distance

% \everymath{\displaystyle} % displays inline math as displaymath

% make title smaller
% \pretitle{\begin{center}\large}
% \posttitle{\end{center}}
\preauthor{\begin{center}\normalsize}
\postauthor{\end{center}}
\predate{\begin{center}\footnotesize}
\postdate{\end{center}}
\setlength{\droptitle}{-60pt}

\title{Assignment 2 - Theoretical Assignment \\
 COMP6741: Parameterized and Exact Computation}
\author{Edward Lee, Han Zhao, Louis Tiao}

%%%%%%%%%%%%%%%%%%%%%%%%%%
% Commonly used theorems %
%%%%%%%%%%%%%%%%%%%%%%%%%%

% Use roman for text - numbering follows onwards
\theoremstyle{definition}
\newtheorem{defn}{Definition}[section]
\newtheorem{them}{Theorem}[section]
\newtheorem{lem}{Lemma}[them]
\newtheorem{prop}[them]{Proposition}
\newtheorem{corr}[them]{Corollary}
\newtheorem{corrr}{Corollary}[them] %for consistency issues
% examples follow different numbering
\newtheorem{eg}{Example}[section]
\newtheorem{egg}[them]{Example}
\newtheorem*{themf}{Theorem 5.0}
\newtheorem*{themm}{Theorem}

%%%%%%%%%%%%%
% Shortcuts %
%%%%%%%%%%%%%

\newcommand{\Q}{\mathbb{Q}}
\newcommand{\C}{\mathbb{C}}
\newcommand{\F}{\mathbb{F}}
\newcommand{\R}{\mathbb{R}}
\newcommand{\Z}{\mathbb{Z}}
\newcommand{\N}{\mathbb{N}}
\newcommand{\D}{\mathbb{D}}

\newcommand{\mcC}{\mathcal{C}}
\newcommand{\mcN}{\mathcal{N}}
\newcommand{\mcE}{\mathcal{E}}
\newcommand{\mcP}{\mathcal{P}}
\newcommand{\mcF}{\mathcal{F}}
\newcommand{\mcS}{\mathcal{S}}
\newcommand{\mcA}{\mathcal{A}}
\newcommand{\mcB}{\mathcal{B}}
\newcommand{\mcX}{\mathcal{X}}
\newcommand{\mcL}{\mathcal{L}}
\newcommand{\mcH}{\mathcal{H}}
\newcommand{\mcl}{\mathcal{l}}
\newcommand{\mcU}{\mathcal{U}}

\newcommand{\mcI}{\mathcal{I}}
\newcommand{\mcO}{\mathcal{O}}

\newcommand{\bfe}{\mathbf{e}}
\newcommand{\mbf}{\mathbf}

\newcommand{\ran}{\mbox{Ran}}
 
\newcommand{\al}{\alpha}
\newcommand{\s}{\sigma}
\newcommand{\ep}{\epsilon}
\newcommand{\dl}{\delta}
\newcommand{\sg}{\sigma}

\newcommand{\bs}{\backslash}
\newcommand{\n}{\\}
\newcommand{\ol}{\overline}

%%%%%%%%%%%%%%%%%%%%%%%%%
% Standard big brackets %
%%%%%%%%%%%%%%%%%%%%%%%%%

\newcommand{\floor}[1]{\left\lfloor{#1}\right\rfloor}
\newcommand{\ceil}[1]{\left\lceil{#1}\right\rceil}
\newcommand{\bbA}[1]{\left({#1}\right)}
\newcommand{\bbB}[1]{\left[{#1}\right]}
\newcommand{\bbC}[1]{\left\{{#1}\right\}}

%%%%%%%%%%%%%%%%%%
% Start Document %
%%%%%%%%%%%%%%%%%%

\begin{document}
\maketitle

\section*{Preliminaries}

For completeness and later reference, we restate the \textsc{Packing}, $d$-\textsc{Packing}, \#\textsc{Packing} and \textsc{Packing Coloring} problems presented in this assignment.

For an integer $d \geq 0$, a \emph{d-packing} in a graph $G = (V,E)$ is a subset of vertices $S \subseteq V$ such that the distance between any two vertices in $S$ is at least $d$ + 1 in $G$. Denote $d(u,v)$ as the distance between $u,v \in V$.

\begin{algorithm}[H]
\textsc{Packing\n}
\begin{tabular}{l l}
 Input:    & A graph $G$, an integer $d \geq 0$, and an integer $k \geq 0$ \n
 Question: & Does $G$ have a $d$-packing of size $k$? \n
\end{tabular}	
\end{algorithm}

For a fixed integer $d$, 

\begin{algorithm}[H]
\textsc{$d$-Packing\n}
\begin{tabular}{l l}
 Input:    & A graph $G$, and an integer $k \geq 0$ \n
 Question: & Does $G$ have a $d$-packing of size $k$? \n
\end{tabular}	
\end{algorithm}

The counting variant requires the output of the number of $d$-packings in $G$

\begin{algorithm}[H]
\textsc{\#Packing\n}
\begin{tabular}{l l}
 Input:    & A graph $G$, and an integer $d \geq 0$\n
 Question: & The number of $d$-packings in $G$ \n
\end{tabular}	
\end{algorithm}

For a sequence of $k$ non-negative integers $D = (d_1,...,d_k)$, a \emph{D-packing coloring} of a graph $G = (V,E)$ is a partition of the vertex set $V$ into $k$ disjoint sets of vertices, $(S_1,...,S_k)$, such that $S_i$ is a $d_i$-packing in $G$ for each $i \in \{1,...,k\}$. The \textsc{Packing Coloring} problem is defined below

\begin{algorithm}[H]
\textsc{Packing Coloring\n}
\begin{tabular}{l l}
 Input:    & A graph $G$, an integer $k \geq 1$, and a sequence of non-negative integers $D = (d_1,...,d_k)$ \n
 Question: & Does $G$ have a $D$-packing coloring? \n
\end{tabular}	
\end{algorithm}

\clearpage

\section{\#\sc Packing}

% \todo{Formalize and Proofread}

Consider a graph $G = (V, E)$ and let $n = |V|$.

First note that in any graph, the distance between any two distinct vertices is at least 1. Therefore, any subset of vertices $S \subseteq V$ is a $0$-packing, of which there are $2^n$. So for $d=0$, we need only calculate $|V|$ and $2^{|V|}$, which can be done in $\mcO(n)$ time and space.

Next observe that in an independent set, the distance between any two distinct vertices is at least 2. Therefore, any independent set of $G$ is a $1$-packing. Denote \texttt{\#Indep-set} the polynomial space and $\mcO^*(1.2377^n)$ time branching algorithm of Wahlstr\"{o}m \cite{Dahllof2005,Wahlstr2008} that counts the number of independent sets in any graph. Then for $d=1$, we simply run \texttt{\#Indep-set} on $G$ using polynomial space, by virtue of the fact that \texttt{\#Indep-set} is a polynomial space branching algorithm.

For $d \geq 2$, we consider the $d$th power $G^d$ of $G$ by adding edges between all pairs of distinct vertices that are of distance distance at most $d$ apart in $G$. Then for any pair of distinct vertices that are not adjacent in $G^d$, the distance between them in $G$ must be at least $d+1$. It immediately follows that a subset of vertices is an independent set of $G^d$ if and only if it is a $d$-packing of $G$.

Let $m=|E|$. Then we compute the graph $G^d$ in time $\mcO(mn)$ and polynomial space by performing a breadth first search starting at each vertex to determine the distances to all other vertices and insert edges where the distance is at most $d$. We then run \texttt{\#Indep-set} on $G^d$ and output its result. Since we have a polynomial time and space reduction and a polynomial space and $\mcO^*(1.2377^n)$ time branching algorithm which solves our reduced instance, we also have a polynomial space and $\mcO^*(1.2377^n)$ time branching algorithm in this case.

We summarize the above in \cref{algo:num_packing} and denote it as \texttt{\#Packing}.

\begin{algorithm}[H]
\caption{\texttt{\#Packing}}          % give the algorithm a caption
\label{algo:num_packing} 
\texttt{\#Packing\n}
\begin{tabular}{l l}
 Input:  & ($G,d$) \n
 Output: & Number of $d$-packings in $G$ \n
\end{tabular}	

\If{$d = 0$}{
	\textbf{return} $2^n$
}
\If{$d = 1$}{
	\textbf{return} \texttt{\#Indep-set}($G$)
}
\If{$d \geq 2$}{
	Let $G^d = (V,E')$ where $E' = E \cup E^*$ and $E^* = \{ uv : d_G(u,v) \leq d \}$\n
	\textbf{return} \texttt{\#Indep-set}($G^d$)
}
\end{algorithm}

Finally, we see that in all cases (i.e. for all $d \ge 0$), \texttt{\#Packing} runs in polynomial space and $\mcO^*(1.2377^n)$ time.

\section{\sc Packing Coloring  (\textnormal{Exponential space})}

First note that a $k$-coloring of a graph is a special case of a $D$-packing coloring 
where $D = (d_1, d_2, \dotsc, d_k) = (1, 1, \dotsc, 1)$. As such, we can consider extending
the solution to the \textsc{Coloring} problem to solve the \textsc{Packing Coloring} problem.

Recall that a $k$-coloring of a graph $G = (V, E)$ corresponds to a partition of $V$ into $k$ independent sets. So a $k$-coloring is a $k$-partition of the set system $(V, \mcI)$ where $\mcI$ is the set of all independent sets of $G$. In fact, obtaining $k$-cover $(I_1, \dotsc, I_k)$ of $(V, \mcI)$ would suffice since we can easily obtain a $k$-partition of $(V, \mcI)$ (and hence $k$-coloring of $G$) by removing $\bigcup_{i=1}^{j-1} I_i$ from $I_j$ for $j \in \{1, \dotsc, k\}$.

Now consider the \textsc{Packing Coloring} problem. Denote $\mcP_i$ as the set of all $d_i$-packings of $G$ and $\mcP = \bigcup_{i=1}^{k} \mcP_i$. Then a graph $G = (V, E)$ has a $(d_1, \dotsc, d_k)$-packing coloring if and only if there exists a $k$-cover $(S_1, \dotsc, S_k)$ of $(V, \mcP)$ such that $S_i$ is a $d_i$-packing. We can obtain $k$-partition (and hence a $D$-packing coloring of $G$) as before, since like independent sets, $d$-packings are closed under set difference.

Let $U$ be the set of tuples $(S_1, \dotsc, S_k)$, where each $S_i$ is a $d_i$-packing and $A_v = \{ (S_1, \dotsc, S_k) \in U : v \in \bigcup_{i=1}^{k} S_i\}$. Clearly $G$ has a $D$-packing coloring if and only if $\left | \bigcap_{v \in V} A_v \right | > 0$.

To compute $\left | \bigcap_{v \in V} A_v \right |$, we use the inclusion-exclusion theorem (c.f. Theorem 4.2 in \cite{Fomin2010}):
\begin{equation} \label{eq:incl_excl}
	\left | \bigcap_{v \in V} A_v \right | = \sum_{S \subseteq V} (-1)^{|S|} \left | \bigcap_{v \in S} \overline{A_v} \right | 
\end{equation}

So by extension, we simply need to compute 
\begin{align*}
	\left | \bigcap_{v \in S} \overline{A_v} \right | &= \left | \{ (S_1, \dotsc, S_k) \in U : S_1, \dotsc, S_k \subseteq V \setminus S \} \right | \\
													  &= \prod_{i=1}^{k} p(V \setminus S, d_i)
\end{align*}

where $p(X, d)$ is the number of $d$-packings in $G[X]$ (the subgraph induced by vertex subset $X$) and can be defined by the following dynamic programming recurrence:
\begin{align*}
	p(\emptyset, d) &= 1 \\
	p(X, d) &= \sum_{v \in X} p(X \setminus {v}, d) + p(X \setminus N_G^d[v], d)   
\end{align*}

(n.b. $N_G^r[v]$ denotes the closed neighborhood of $v$ of radius $r$, that is, $u \in N_G^r[v]$ iff $d(u, v) \leq r$.) The table for $p(X, d)$ is filled by increasing cardinalities of $X$ and therefore we compute it in $\mcO^*(2^n)$ time and space. 

Finally, we evaluate

\begin{equation} \label{eq:incl_excl_final}
	\left | \bigcap_{v \in V} A_v \right | 
	= \sum_{S \subseteq V} (-1)^{|S|} \left | \bigcap_{v \in S} \overline{A_v} \right | 
	= \sum_{S \subseteq V} (-1)^{|S|} \prod_{i=1}^{k} p(V \setminus S, d_i)
\end{equation}

in $\mcO^*(2^n)$ time and space. We can decide whether $G$ has a $D$-packing coloring by checking that $G$ has at least one $D$-packing coloring ($\left | \bigcap_{v \in V} A_v \right | > 0$). Therefore, \textsc{Packing Coloring} can be solved in $\mcO^*(2^n)$ time and space.

\section{\sc Packing Coloring (\textnormal{Polynomial space})}

We incorporate the algorithm \texttt{\#Packing} from (1) that counts the number of $d$-packings in a given graph $G$ and integer $d \ge 0$ in $\mcO^*(1.2377^n)$ time and polynomial space to modify \cref{eq:incl_excl_final} to be
\begin{equation*}
	\left | \bigcap_{v \in V} A_v \right | 
	= \sum_{S \subseteq V} (-1)^{|S|} \prod_{i=1}^{k} \texttt{\#Packing}(V \setminus S, d_i)
\end{equation*}

Then, we obtain a polynomial space algorithm for \textsc{Packing Coloring} with running time
\begin{equation*}
	\sum_{S\subseteq V} \mcO^*(1.2377^{n-|S|}) = \sum_{s=0}^n \binom{n}{s} \mcO^*(1.2377^{n-s}) = \mcO^*(2.2377^n)
\end{equation*}

Therefore, \textsc{Packing Coloring} can be solved in $\mcO^*(2.2377^n)$ time and polynomial space.

\section{\sc Packing \textnormal{is FPT for parameter $k + d + \Delta (G)$} }

We restrict our attention to \emph{maximal} $d$-packings. A $d$-packing is maximal if it is not a proper subset of any other $d$-packing. Observe that for any vertex $v$ of $G$, every maximal $d$-packing contains a vertex from $N_G^d[v]$: if a vertex $v$ is in $d$-packing $S$ then no vertex in its neighborhood of radius $d$ can be in $S$. On the other hand, if $S$ is a maximal $d$-packing and $v \notin S$ then at least one vertex in its neighborhood of radius $d$ must be in $S$ since, otherwise, $S\cup \{ v \}$ would be a larger $d$-packing, contradicting the optimality of $S$.

So for a graph $G = (V, E)$, we select any vertex $v \in V$ and branch on each vertex $u$ in its neighborhood of radius $d$ $N_G^d[v]$, recursively checking whether $G-N_G^d[v]$ is a $d$-packing of size $k-1$. 

Abbreviate $\Delta = \Delta(G)$. We have a $|N_G^d[v]|$-way branching for vertex $v \in V$. Since $|N_G^d[v]| \leq \sum_{i=0}^{d} \Delta^i = \frac{\Delta^{d+1}-1}{\Delta-1} = \mcO(\Delta^{d+1})$ and as $k$ decreases by 1 in each branch, we obtain a running time of $\mcO^{*}((\Delta^{d+1})^k)$ since we only require $n^{\mcO(1)}$ in each node of the recursive call tree. The algorithm is summmarized in \cref{algo:packing}.

\begin{algorithm}[H]
\caption{\texttt{Packing}}          % give the algorithm a caption
\label{algo:packing} 
\texttt{Packing}\n
\begin{tabular}{l l}
 Input:  & ($G, d, k$) \n
 Output: & \texttt{true} if $G$ has $d$-packings of size $k$; \texttt{false} otherwise. \n
\end{tabular}	

\If{$k = 0$}{
	\textbf{return} \texttt{true}
}
\ElseIf{$k > |V|$}{
	\textbf{return} \texttt{false}
}
\Else{
	Select a vertex $v \in V$.

	\For{$u \in N_G^d[v]$}{	
		\If {\texttt{Packing}($G-N_G^d[u], d, k-1$)} {
			\textbf{return} \texttt{true}
		}
	}
}
\end{algorithm}

Since \texttt{Packing} is a parameterized branching algorithm with running time $(\Delta^{d+1})^k \cdot n^{\mcO^(1)}$, \textsc{Packing} is FPT for parameter $k + d + \Delta (G)$.

\clearpage
\section{$d$-\sc Packing \textnormal{is $W[1]$-hard for parameter $k$}}

% \todo{Formalize and Proofread}

We show that the $d$-\textsc{Packing} problem is $W[1]$-hard for parameter $k$ by providing a parameterized reduction from a well-known $W[1]$-hard problem, namely, \textsc{Independent Set}.	For $d = 0$ and $d = 1$ the reduction is trivial as we don't alter the input graph $G$ and $k$. So we work on $d \geq 2$.

\textbf{Construction:} Let $(G = (V,E) ,k)$ be an instance for $\textsc{Independent Set}$. We construct an instance $(G' = (V',E'),k')$ for $d-$\textsc{Packing}. First set $k' \leftarrow k$. 

For odd $d = 2m+1$ where $m$ is a natural number, for every vertex $v \in V$ we append a path of length $m$ to $v$. In other words we add $m$ vertices $v_1, v_2, ..., v_m \in V'$ and $vv_1, v_1v_2,...,v_{m-1}v_m \in E'$ for every vertex $v$.
For even $d = 2m$, where $m > 1$ we add a path graph of $m$ vertices as described for the odd case of $d = 2m - 1$. Then we also add a vertex $w_{uv} \in V'$ if $uv \in E$.
The last case is for $d = 2$ where we add a vertex $w_{uv} \in V'$ if $uv \in E$ and also add an edge between all these new added vertices, that is to say $w_{i,j}w_{k,l} \in E'$ if $w_{i,j},w_{k,l} \in V'$ and $ij \in E$ and $kl \in E$.

% we first construct the square of $G^2$ obtained by taking $G$ and adding edges between all pairs of distinct vertices that are of distance at most 2 apart in $G$. Then we add a path graph of $m$ vertices as described for the odd case $d + 1 = 2m+1$ to every vertex $v \in V$ onto the square graph $G^2$.

\begin{figure}[ht]
\centering
\begin{minipage}{.5\textwidth}
  	\centering
  	\begin{tikzpicture}[shorten >= 1pt, node distance=1cm, auto, inner sep=0pt]
		\node[state, minimum size = 20pt] (1) {};
		\node[state, minimum size = 20pt] (2) [below=of 1] {$v$};
		\node[state, minimum size = 20pt] (3) [right=of 1] {};
		\node[state, minimum size = 20pt] (4) [above left=of 1] {$u$};

		\path[] (1) edge (2);
		\path[] (1) edge (3);
		\path[] (1) edge (4);
		\path[] (2) edge (3);
		
	\end{tikzpicture}
	\caption{Example graph $G$}
\end{minipage}%
\begin{minipage}{.5\textwidth}
	\centering
	\begin{tikzpicture}[shorten >= 1pt, node distance=1cm, auto, inner sep=0pt]
		\node[state, minimum size = 20pt] (1) {};
		\node[state, minimum size = 20pt] (2) [below=of 1] {$v$};
		\node[state, minimum size = 20pt] (3) [right=of 1] {};
		\node[state, minimum size = 20pt] (4) [above left=of 1] {$u$};

		\node[state, minimum size = 10pt] (5) [above=of 4] {};
		\node[state, minimum size = 10pt] (6) [above=of 5] {$u_2$};
		\node[state, minimum size = 10pt] (7) [above=of 1] {};
		\node[state, minimum size = 10pt] (8) [above=of 7] {};
		\node[state, minimum size = 10pt] (9) [right=of 2] {};
		\node[state, minimum size = 10pt] (10) [right=of 9] {$v_2$};
		\node[state, minimum size = 10pt] (11) [right=of 3] {};
		\node[state, minimum size = 10pt] (12) [right=of 11] {};

		\path[] (1) edge (2);
		\path[] (1) edge (3);
		\path[] (1) edge (4);
		\path[] (2) edge (3);

		\path[dashed] (4) edge (5);
		\path[dashed] (5) edge (6);
		\path[dashed] (1) edge (7);
		\path[dashed] (7) edge (8);
		\path[dashed] (2) edge (9);
		\path[dashed] (9) edge (10);
		\path[dashed] (3) edge (11);
		\path[dashed] (11) edge (12);
		
	\end{tikzpicture}
	\caption{$G'$ construction for $d = 5$}
\end{minipage}
\end{figure}

\begin{figure}[ht]
\centering
\begin{minipage}{.5\textwidth}
	\centering
	\begin{tikzpicture}[shorten >= 1pt, node distance=2cm, auto, inner sep=0pt]
		\node[state, minimum size = 20pt] (1) {};
		\node[state, minimum size = 20pt] (2) [below=of 1] {$v$};
		\node[state, minimum size = 20pt] (3) [right=of 1] {};
		\node[state, minimum size = 20pt] (4) [above left=of 1] {$u$};
		\draw [black, fill] (-1,1) circle [radius = 0.15];
		\draw [black, fill] (1.4,0) circle [radius = 0.15];
		\draw [black, fill] (0,-1.4) circle [radius = 0.15];
		\draw [black, fill] (1.35,-1.35) circle [radius = 0.15];

		\node[state, minimum size = 10pt] (5) [above=of 4] {$u_1$};
		% \node[state, minimum size = 10pt] (6) [above=of 5] {$u_2$};
		\node[state, minimum size = 10pt] (7) [above=of 1] {};
		% \node[state, minimum size = 10pt] (8) [above=of 7] {};
		\node[state, minimum size = 10pt] (9) [right=of 2] {$v_1$};
		% \node[state, minimum size = 10pt] (10) [right=of 9] {$v_2$};
		\node[state, minimum size = 10pt] (11) [right=of 3] {};
		% \node[state, minimum size = 10pt] (12) [right=of 11] {};

		\path[] (1) edge (2);
		\path[] (1) edge (3);
		\path[] (1) edge (4);
		\path[] (2) edge (3);

		\path[dashed] (4) edge (5);
		% \path[dashed] (5) edge (6);
		\path[dashed] (1) edge (7);
		% \path[dashed] (7) edge (8);
		\path[dashed] (2) edge (9);
		% \path[dashed] (9) edge (10);
		\path[dashed] (3) edge (11);
		% \path[dashed] (11) edge (12);
		
	\end{tikzpicture}
	\caption{$G'$ construction for $d = 4$}
\end{minipage}%
\begin{minipage}{.5\textwidth}
	\centering
	\begin{tikzpicture}[shorten >= 1pt, node distance=1cm, auto, inner sep=0pt]
		\node[state, minimum size = 20pt] (1) {};

		\node[state, minimum size = 10pt] (2) [below=of 1] {};
		\node[state, minimum size = 20pt] (3) [below=of 2] {$v$};


		\node[state, minimum size = 10pt] (4) [right=of 1] {};
		\node[state, minimum size = 20pt] (5) [right=of 4] {};
		
		\node[state, minimum size = 10pt] (6) [above left=of 1] {};
		\node[state, minimum size = 20pt] (7) [above left=of 6] {$u$};

		\node[state, minimum size = 10pt] (8) [above right=of 3] {};


		\path[] (1) edge (2);
		\path[] (1) edge (4);
		\path[] (1) edge (6);
		\path[] (2) edge (3);
		\path[] (4) edge (5);
		\path[] (6) edge (7);
		\path[] (3) edge (8);
		\path[] (5) edge (8);

		\path[dashed] (2) edge (4);
		\path[dashed] (2) edge (6);
		\path[dashed] (2) edge (8);
		\path[dashed] (4) edge (8);
		\path[dashed] (4) edge (6);
		\path[dashed, bend right = 90] (6) edge (8);
		
		

		
	\end{tikzpicture}
	\caption{$G'$ construction for $d = 2$}
\end{minipage}
\end{figure}

\clearpage

\textbf{Equivalence}: We will show that $G$ has an independent set of size $k$ iff $G'$ has a $d$-packing of size $k$.  In all following proofs, $m \in \N$ and $m \geq 1$. Furthermore, we may assume that both $G$ and $G'$ do not contain isolated vertices since if $x$ is isolated then $(G,k)$ reduces to $(G\bs\{x\},k-1)$ in both instances. 

$(\Rightarrow):$ Suppose that $S \subseteq V$ is an independent set of size $k$ in $G$.

\textbf{Odd Case:} For odd $d = 2m+1$, consider an arbitrary vertex $s \in S$. After the construction of $G'$, let $s_m$ denote the $m$th vertex at the end of the path added to $s$. (We also remark that $\mathrm{deg}_{G'}(s_m) = 1$ as it is a pendant vertex.) Then consider the set
\[ S' = \{ s_m : s_m \in V' \text{ corresponding to } s \in S \}\]
which has size $k$ since $|S| = k$.
Since $S$ is an independent set, then for two arbitrary vertices $u,v \in S$, $d(u,v) \geq 2$. Due to the construction, then $d(u_m,u) = d(v_m,v) = m$ in $G'$. Hence $d(u_m,v_m) = d(u_m,u) + d(u,v) + d(v,v_m) \geq 2m + 2$ in $G'$. As this is true of any two vertices $u_m, v_m \in S'$, the shortest path between any two vertices of $S'$ is $2m+2 = d+ 1$, making $S'$ a $d$-packing of size $k$.

\textbf{Even Case:} For even $d = 2m$, $m > 1$ the proof is similar to the odd case for $d = 2m - 1 = 2(m-1) + 1$, including the construction of $S'$ (and $|S'| = k$) except that we only add in paths of length $m-1$ and hence we only add $m-1$ new vertices for the path graph, for each vertex $v \in V$. Now due to the construction, for two arbitrary vertices $u,v \in S$ $d(u,v) \geq 4$ in $G'$ as each previous edge of $G$ has been replaced by two edges in $G'$. The corresponding elements $u_{m-1}, v_{m-1} \in S'$ are gained by taking the pendant vertex at the end of the $m-1$ length path graph stemming from $u$ and $v$ respectively. Now $d(u_{m-1},v_{m-1}) = d(u_{m-1},u) + d(u,v) + d(v,v_{m-1}) \geq (m - 1) + 4 + (m-1) = 2m + 2 > 2m + 1$, hence any two vertices in $S'$ are of distance greater than or equal to $2m + 1 = d+1$ meaning S' is a $d$ packing.

\textbf{2-Case:} For $d = 2$, let $S' = S$. Consider two vertices $u,v \in S'$. Let $u' \in V'$ be the closest middle-vertex to $u$, one which was added in the construction of $G'$ for $d=2$. Similarly denote $v'$ for a vertex with the same properties for $v$. Both $u'$ and $v'$ exist since there are no isolated vertices. Now $d(v',u') = 1$ by construction and the fact that $u' \neq v'$ since $uv \notin E$. This means that $d(u,u') = d(v,v') = 1$ as they are the closest middle-vertex and $d(u,v) = d(u,u') + d(u',v') + d(v',v) = 3$, for any vertex $u,v \in S'$. Thus $S'$ is a 2-packing as any two vertices in $S'$ are of distance 3 or greater from each other.

% Now this means that we have a lower bound on the distance $d(u_m,v_m) = d(u_m,u) + d(u,v) + d(v_m,v) \geq 2m + 1$. But as this indicates that for any two vertices $S'$ are at least a distance $2m + 1 = d+1$ away meaning $S'$ is a $d$-packing of size $k$.


$(\Leftarrow):$ Suppose that $S' \subseteq V$ is a $d$-packing of size $k$ in $G' = (V', E')$. Notice that $G'$ was constructed from $G$ so let $V \subseteq V'$ denote those vertices in $G'$ which were originally in $G$.

\textbf{Odd Case:} For odd $d = 2m+1$, construct a set $S = \{s_1, s_2, ..., s_m\}$ from $S' = \{s'_1, s'_2, ... ,s'_k\}$ by
\[ s_i = \begin{cases}
	s'_i, \text{ if } s'_i \in V & \n
	v, \text{ if }  s'_i = v_j & \text{for some } v_j \in V' \text{ with $v \in V$ being the start of} \n & \text{the path graph which $v_j$ lies on}
\end{cases}\]
In other words, we obtain $S$ by taking the elements of $S'$ and mapping them pack down to a vertex in $G$ by following the path graph, which was added to construct $G'$. Once again, let us consider two vertices $u,v \in S'$. We have 3 cases to deal with:
\begin{enumerate}
	\item $u,v \in V$ - This means that both vertices in the $d$-packing already lie on $G$. By definition of $S$, $u$ and $v$ are also in $S$. Since both $u,v \in S'$ then $d(u,v) \geq d+1$ in $G$ and in $G'$ which means that they cannot be adjacent in $G$.
	\item $u \in V$, $v = v_j$ for some added vertex $v_j$ on an added path - Let $v' \in V$ be the vertex which the path containing $v_j$ starts from due to the construction of $G'$. In other words, $v'$ is the vertex from the original graph and $v$ is a path-vertex of the constructed path length $m$ starting from $v'$. By definition of $S$, we place $u$ and $v'$ in $S$. Now $d(v,v') \leq m$ by construction of $G'$. Since $d(u,v') = d(u,v) - d(v,v') \geq d + 1 - m = m + 1$, meaning that $u$ and $v'$ are not adjacent in $G$.
	\item $u = u_i$ and $v = v_j$ for some added vertex $u_j,v_j$ by construction of $G'$ -  Let $u' \in V$ and $v' \in V$ be the vertex in $G$ from which the path containing $u$ and $v$ start from, respectively. By definition of $S$, $u'$ and $v'$ are both in $S$. Now $d(u',v') = d(u,v) - d(u,u') - d(v,v')$ since the shortest path from $u$ to $v$ must path through both $u'$ and then $v'$ by the construction of $G'$. Since $d(u,u') \leq m$ and $d(v,v') \leq m$ then $d(u',v') \geq d + 1 - m - m = 2$. Hence $u', v'$ are not adjacent in $G$
\end{enumerate}
These three cases show that for any two vertices $u,v \in S'$, they get mapped onto two vertices in $S$ such that those two vertices are not adjacent to each other. As this applies for all pairs, then $S$ is an independent set since no two vertices share an edge.

\textbf{Even Case:} For even $d = 2m$, we apply the same procedure as we would for odd $d = 2(m-1) + 1$. The construction of $S = \{s_1,s_2,...,s_k\}$ from $S' = \{s'_1,s'_2,...,s'_k\}$ has a new addition for the new added vertices, one for each edge of $G$.
\[ s_i = \begin{cases}
	s'_i, \text{ if } s'_i \in V & \n
	v, \text{ if }  s'_i = v_j & \text{for some } v_j \in V' \text{ with $v \in V$ being the start of} \n & \text{the path graph which $v_j$ lies on} \n
	u, \text{ if } & s'_i \text{ is a middle vertex and $u$ is one of the 2 vertices adjacent to $s'_i$}. \n
	& \text{i.e. } us'_i \in E' \n
\end{cases}\]

So we take two vertices $u,v \in S'$ and consider 6 cases. The addition we need to make is that now $u$ or $v$ could be a \emph{middle-vertex}, a vertex which is added to the edges of the graph $G'$ in the even construction.

\begin{enumerate}
	\item $u,v \in V$ - As with the odd case, this means by definition of $S$ then both $u,v \in S$ lie on the original graph $G$ with $d(u,v) \geq d + 1 = 2m + 1$ in $G'$, and thus $d(u,v) \geq m$ in $G$, due to the double of edges in the construction of $G'$. Since $m > 1$ then $u,v$ are not adjacent in $G$.

	\item $u,v$ are both a \emph{middle-vertex} - Let $u',v' \in V$ be vertices adjacent to $u$ and $v$ respectively. Now $u' \neq v'$ since $d(u,v) \geq 2m + 2$ as $m>1$ and $d(u,v)$ must be even since the path between two \emph{middle-vertex} alternates between a middle-vertex and an original vertex (from $V$). Then since $d(u',v')$ is even by a similar argument, then as $d(u,u') = 1$ and $d(v,v') = 1$, we have $d(u',v') \geq d(u,v) - d(u,u') - d(v,v') \geq 2m + 2 - 2  = 2m$ in $G'$. This means that $d(u',v') \geq m$ in $G$ as every edge in $G$ was doubled to make an edge in $G'$, and since $m > 1$ then $u',v' \in S$ are not adjacent in $G$.  

	\item $u = u_i$ and $ v = v_j$ for some vertex on the added path of length $m-1$. Denote $u'$ and $v'$ as the vertex in $G$ from which the path containing $u$ and $v$ start from respectively. By construction of $S$, both $u'$ and $v'$ are in $S$. Now since $d(u,u') \leq m-1$, $d(v,v') \leq m-1$ and $d(u,v) \geq 2m + 1$ then $d(u',v') = d(u,v) - d(u,u') - d(v,v') \geq 2m + 1 - (m-1) - (m-1) \geq 3$ in $G'$. We have argued before that $d(u',v')$ is an even number since $u'$ and $v'$ lie on the original graph $G$, and the construction of $G'$ doubles those edges. As $d(u',v') \geq 4$ in $G'$ means $d(u',v') \geq 2$ in G then $u',v' \in S$ are not adjacent.

	\item $u \in V$ and $v = v_j$ for some vertex on the added path. We look for $v' \in V$ such that $v$ lies on the path graph of length $m-1$ extended from vertex $v'$. By construction of $S$ we get that $u,v' \in S$. Now since $d(u,v) \geq 2m+1$ and $d(v,v') \leq m - 1$ then $d(u,v') = d(u,v) - d(v,v') \geq 2m + 1 - (m-1) = m + 2 \geq 3$ in $G'$. As $u,v'$ are both in $V$ then by the even length counting argument we had above, $d(u,v') \geq 4$ in $G'$ and hence $d(u,v') \geq 2$ in $G$ meaning $u,v'$ are not adjacent in $G$.

	\item $u \in V$ and $v$ is a \emph{middle-vertex} - Denote $v' \in V$ to be one of the two possible vertices in $G$ that $v$ is adjacent to (WLOG we pick one of them and assume worst path bounds). By construction of $S$ from $S'$ we get that $v' \in S$. Then $d(v,v') = 1$ and $d(u,v) \geq 2m+1$ meaning that $d(u,v') \geq d(u,v) - d(v,v') \geq 2m + 1 -1 = 2m$ in $G'$. Since there are twice as many edges in $G'$ by construction then $d(u,v') \geq m$ in $G$ but as $m > 1$ then $u,v' \in S$ are not adjacent in $G$.  

	\item $u = u_i$ and $v$ is a \emph{middle-vertex} - Denote $u' \in V$ to be the vertex in $G$ from which the path of length $m-1$ containing $u$ starts from. Denote $v' \in V$ to be a vertex adjacent to $v$, of which there is guaranteed to be one. $u',v' \in S$ by construction. Now $d(u,v) \geq 2m+1$, $d(u,u') \leq m-1$ and $d(v,v') = 1$. Then $d(u',v') \geq d(u,v) - d(u,u') - d(v,v') = 2m+1 - (m-1) - 1 = m + 1 \geq 3$ in $G'$. We have seen above that the fact that $u',v' \in V$ then $d(u',v')$ is even hence $d(u',v') \geq 4$ in $G'$. As $G'$ doubles edges in $G$ by construction then $d(u',v') \geq 2$ in $G$.
\end{enumerate}

These 6 cases cover all possible configurations for any two vertices $u,v \in S'$. As we have shown, by following the construction of $S$, in all cases, any two vertices in $S$ will not be adjacent in $G$. Hence $S$ is an independent set.

\textbf{2-Case:} Lastly, for $d = 2$, we consider two vertices $u,v \in S'$ and attempt to characterize them. Let the vertices added in the construction of $G'$ for $d = 1$ be denoted as \emph{middle-vertices}. There are three cases:

\begin{enumerate}
	\item $u,v \in V$
	\item $u \in V$, $v$ is a \emph{middle-vertex}
	\item $u$ and $v$ are \emph{middle vertices}.
\end{enumerate}

Now since $u,v \in S'$ which is a 2-packing then $d(u,v) \geq 3$. Now in Case 3, by construction $d(u,v) = 1$ since all middle vertices are adjacent to all other middle vertices, hence $u,v$ cannot satisfy case 3. 

In Case 2, there exists a \emph{middle-vertex} $u'$ which is adjacent to $u$ since there are no isolated vertices. That is to say, $d(u,u') = 1$. Now $d(u',v) = 1$ since both $u'$ and $v$ are \emph{middle-vertices} meaning $d(u,v) = 2$. Hence $u,v$ cannot satisfy Case 2.

This means that any two vertices $u,v \in S'$ must satisfy case 1.

An independent set $S = \{s_1,s_2,...,s_k\}$ is constructed from $S' = \{s_1,s_2,...,s_k\}$, that is to say that a 2-packing on $G'$ corresponds directly to an independent set. This is because we have shown that $S'$ only contains vertices in $V$, the original vertex set of $G$ and the fact that if $u,v$ were in fact adjacent then $d(u,v) = 2$ in $G'$ which contradicts $d(u,v) \geq 3$. Thus $S$ is an independent set for $G$.

Now as we have exhausted all cases, then any \textsc{Yes} instance of \textsc{Independent Set} corresponds to a \textsc{Yes} instance of $d-$\textsc{Packing} for any fixed $d$. 

% Since $k' = k$ then it is upper bounded by the identity function. The running time of the algorithm is polynomial as $d$ is fixed. Hence we have the reduction 
% \[ \text{\textsc{Independent Set}} \leq_{\text{FPT}}\text{$d$-\textsc{Packing}} \] 
% By virtue of the fact that \textsc{Independent Set} is W[1]-hard and there is a FPT reduction to $d$-\textsc{Packing} then $d$-\textsc{Packing} is also W[1]-hard for parameter $k$.

The reduction is clearly FPT (even polynomial) since $k' \leq k$ and the running time of the reduction is $f(k) \cdot |I|^{\mcO(1)} = n^{\mcO(1)}$ since $f(k)=c$ and $|I| = n$.

Since there is a parameterized reduction from \textsc{Independent Set}, the $d$-\textsc{Packing} problem is $W[1]$-hard for parameter $k$. 

% Now notice that since every vertex is next to a \emph{middle-vertex}, we can let $u'$ and $v'$ be a middle-vertex next to $u,v$ respectively. Now $u' \neq v'$ since otherwise that would mean $d(u,v) = 2$, a contradiction. 


% For even $d = 2m$, the definition of $S$ is exactly the same and the proof is very similar, except since we now have that $G'$ was constructed on $G^2$ we have some new bounds on 

% In other words, given any instance $(G=(V,E), k)$ of \textsc{Independent Set}, we describe an FPT algorithm that contructs an equivalent instance $(G'=(V',E'), k')$ of $d$-\textsc{Packing}.

% First, set $k' \leftarrow k$. For odd $d=2m+1 \ge 1$ where $m$ is some natural number, we replace the edge between all pairs of distinct vertices $(u, v)$ by a path graph on $m$ vertices ($P_m$) by connecting each respective endpoint of the path graph to a vertex $u, v$. In other words, we add $m$ vertices $w_1, w_2, \dotsc, w_m \in V'$ for each pair $(u, v)$ of adjacent vertices, remove the original edge $uv$, and add edges $uw_1, w_1w_2, \dotsc, w_mv \in E'$.

% For even $d=2m$, we first add $m$ vertices for all pairs of adjacent vertices as before. For edge $i$ denote its added vertices $V_{i} = \{w_{i1}, w_{i2}, \dotsc, w_{im}\}$ for $i \in \{1, 2, \dotsc, |E|\}$ and let $V' = V \cup V^{*}$ where $V^{*} = \bigcup_{i=1}^{|E|} V_i$. Then for each pair of vertices $(u, v) \in V_i \times V_j$ where $i \ne j$, we add a path graph on $2m-2 = d-2$ vertices. In other words, we add a path with $d-2$ vertices between vertices $w_{ip}$ and $w_{jq}$ where $i\ne j$, and hence a path of length $d-1$. Let us call the vertices added between pairs of adjacent vertices in $G$ (where we added $m$ vertices) the First Addition, and the vertices added between all pairs of First Addition vertices (in the new $d-1$ vertices) the Second Addition.

% To help aid our proof, we make the assumption that both $G'$ and $G$ do not contain isolated vertices. This case can easily be taken care of since if in $(G,k)$, $x$ is an isolated vertex then both \textsc{Packing} and \textsc{Independent Set} have the same simplification of $(G\bs\{x\},k-1)$.

% Now we show that $(G, k)$ is a \textsc{Yes}-instance for \textsc{Independent Set} iff $(G', k')$ is a \textsc{Yes}-instance for $d$-\textsc{Packing}:

% \begin{itemize}
% 	\item ($\Rightarrow$) Let $S$ be an independent set of size $k$ in $G$. Then, for odd $d$, the distance between any two distinct vertices $(u, v)$ in $S$ in $G'$ is at least $2(m+1)=d+1$ since $uv \notin E$. So if there was one vertex on a path between $u, v$ in $G$, then there are now $d+1$ vertices on a path between them in $G'$. Therefore, $S$ is a $d$-packing of size $k'$ in $G'$ for some odd $d$. For even $d$, \dots

% 	% For even $d$,  if two vertices were originally not neighbours, they are now of distance $d-1 + 2 = d+1$ apart. Hence, $S$ is a $d$-\textsc{Packing} of $G'$ since the distance between any pair of the vertices of $S$ in $G'$ is at least $d+1$.
% 	\item ($\Leftarrow$) Let us first consider the even case for $d = 2m$. Suppose $S$ is a $d-$packing of size $k'$ in $G'$. For any two arbitrary vertices $u,v$ we consider in what cases they can both belong to $S$
% 	\begin{enumerate}
% 		\item $u,v \in V$: The two vertices are from the original graph $G$. If $u$ and $v$ were adjacent to each other in $G$ then notice that the distance from $u$ to $v$ is now $m+1 < 2m+1$ since we added replaced edges with $m$ vertices on a path. In other words, both $u,v$ cannot be in $S$ if they were previously adjacent
% 		\item $u \in V$, $v \in$ First Addition: $u$ belongs to a vertex in $G$ and $v$ is one of the $m$ vertices added in the First Addition. Due to the vertices of the Second Addition, there is a path from $v$ to any $v' \in$ First Addition of path length $d-1$. Let $v'$ be a First Addition vertex adjacent to $u$. Such a $v'$ must exist due to the assumption that we have no isolated vertices. But that means that there exists a path of length $d-1 + 1$ (from $v$ to $v'$ to $u$) between $u,v$ hence both $u,v$ cannot be in $S$.
% 		\item $u \in V$, $v \in $ Second Addition: This works a similar manner to above. There must exist a vertex $v' \in $ First Addition such that $u, v'$ are adjacent. $v$ and $v'$ are at most $d-2 / 2 < d - 1$ distance away. The addition of these two paths will not result in a path greater than $d$ hence once again, both $u,v$ cannot be in $S$
% 		\item $u,v \in$ First Addition: By the addition of the $d-1$ length path by Second Addition, this case cannot occur.
% 		\item $u \in$ First Addition, $v \in$ Second Addition.
% 	\end{enumerate}
% \end{itemize}



\section{\sc Packing \textnormal{is FPT for parameter $d + tw(G)$}}

We formalize the property of a graph having a $d$-packing of size $k$ in Monadic Second Order Logic (MSOL) using a formula of length $\mcO(d)$,
\begin{align*}
	\mathrm{path}(s, t, d) 		&\coloneqq \exists v_0 \in V, \exists v_1 \in V, \dotsc, \exists v_d \in V
									\left [ (s=v_0) \wedge (t=v_d) \right.\nonumber \\
								&\qquad \left. \wedge (adj(v_0, v_1) \wedge adj(v_1, v_2) 
									\wedge \dots \wedge adj(v_{d-1}, v_d)) \right ] \\
	\mathrm{isPacking}(X, d)	&\coloneqq \neg \left [\exists u \in X, \exists v \in X (\mathrm{path}(u, v, 1) 
									\vee \mathrm{path}(u, v, 2) \vee \dots \vee \mathrm{path}(u, v, d) ) \right ]
\end{align*}

In Arnborg et al.'s \cite{Arnborg1991} extension of Courcelle's Theorem gives us an FPT algorithm which takes as an input a graph $G$ and a Monadic Second Order sentence $F(X)$ and computes a minimize size set of vertices $X$ such that $F(X)$ is true in $G$. Here the parameter used to calculate the parameterized complexity is $|F(X)| + tw(G)$. 

In \textsc{Packing} we return whether or not a $d-$packing of size $k$ on a graph $G$ with treewidth $tw(G)$ exist. This can be achieved by acquiring $X$, the minimize size of a a $d-$packing in a graph $G$ with treewidth $tw(G)$ by Arnborg's FPT extension of Courcelle's Theorem using the MSOL sentence isPacking$(G,d)$. Once this minimum size $d-$packing $X$ is acquired, then a simple check if $|X| \leq k$ will ensure that a $d$-packing of size $k$ exists for $G$. If $|X| > k$ then $G$ does not have a $k$ sized $d-$packing.

% \todo{Discussion: Set cardinality is technically not a part of MSOL. Of course we can define it, but the length of the formula is $\mcO(k^2)$.}

% By Courcelle's Theorem, deciding whether a graph satisfies the properties defined by a MSOL formula is FPT, parameterized by the treewidth of $G$ and the length of the formula, which in our case is $\mcO(d)$.

The usage of Arnborg's FPT extension of Courcelle's theorem with the MSOL sentence isPacking$(G,d)$ guarantees that finding a satisfying minimum size graph $X$ is FPT, parameterized by $tw(G) + |$isPacking$(G,d)| = tw(G) + d$. As checking if $|X| \leq k$ is a constant time operation therefore it immediately follows that \textsc{Packing} is FPT for parameter $d+tw(G)$ where $tw(G)$ denotes the treewidth of $G$.

\bibliographystyle{plain}
\bibliography{asst2}

\todo{Thanks to mendeley, the references are actually kinda shit. fix this up if we have time}

\end{document}